\documentclass[11pt, oneside]{article}   	% use "amsart" instead of "article" for AMSLaTeX format
\usepackage{geometry}                		% See geometry.pdf to learn the layout options. There are lots.
\geometry{letterpaper}           
\usepackage[parfill]{parskip} 
\usepackage{graphicx}	
\usepackage{amssymb}
\usepackage{amsfonts}
\usepackage{amsmath}
\usepackage[latin1]{inputenc}
\usepackage[norsk]{babel}

%\newcommand*{\R}{\mathbb{R}}
%\newcommand*{\abs}[1]{{\lvert#1\rvert}}
%\newcommand*{\Abs}[1]{{\Big\lvert#1\Big\rvert}}
%\newcommand*{\ABS}[1]{{\left\lvert#1\right\rvert}}
%\newcommand*{\set}[2]{{\{#1 \mid #2\}}}

\title{Gravitaional N-body Simulations}
\author{Gudbrand Tandberg}
%\date{}							% Activate to display a given date or no date

\begin{document}
\maketitle

\section{Introduksjon til N-legeme problemet}

\begin{itemize}
\item Historikk og matematikk
\item Direkte metoder
\item Skalering; utleder uttrykk for karakteristisk tid og lengde
\end{itemize}

\section{Utledning av diverse l�sningsalgoritmer}
Skisserer kort ODESolver-hiearkiet, for � forklare at kun advance()-metoden trenger � endres p�.

\begin{itemize}
\item RK4
\item Verlet
\item Monte Carlo
\end{itemize}

\section{En komplett generell objektorientert N-body l�ser med konstant timestep.}

Dette blir utgangspunktet for alt som f�lger - utvides i senere deler. Kanskje det aller viktigste av alt er at denne (og ellers struktureringen av programmet) er tenkt ut til minste detalj f�r implementering.

\begin{itemize}
\item 3 forskjellige l�sningsmetoder som egne subklasser: RK4, Verlet, Monte Carlo.  
\item F�rste anvendelse: solsystemet med satelitter (og kankjse Kuiperbeltet!) - lag fin animasjon og analys�r totalenergi-utvikling, spinn-utvikling, stabilitet osv..
\end{itemize}

\section{En ny komplett generell objektorientert N-body l�ser med adaptiv timestep og �n l�sningsmetode (Force Polynomial).}

\begin{itemize}
\item Adaptive, quantized (block time step $\Delta t_n = \Delta t_0/2^n$) timestep
\item God l�sningsmetode - Force Polynomial.
\item det er denne som utvides i del 5
\end{itemize}

\section{Videre mulige utviklinger}
\subsection{Close encounter regularization}
Det finnes standard metoder for � unng� slike singulariteter. Sikkert helt overkommelig � implementere. (kanskje motivere med et d�rlig close encounter eksempel - jorden blir �delagt av Jupiter...)

\subsection{Parallellisering}
Det er jo pensum, s� det burde gj�res. Dessuten tenker jeg � ta 'Parallellprogrammering for naturvitenskapelige andvendelser' neste �r, s� det er jo lurt. 

\subsection{Feilanalyse}
Her kan matematikeren i meg kose seg litt med Taylorutvikling.

\subsection{Stjernecluster simulering}
Som en slags finale. Pr�ve � la $N\sim10^4$ med forskjellige interresante startposisjoner.

\subsection{Grafikk}
Hadde v�rt kult � f� det til � se litt fint ut ogs�.. 3D, sort bakgrunn osv. 

\section{Ting jeg kan se litt p�}
\begin{itemize}
\item Barnes-Hut Treecode - mange som bruker den (og parallelliserer)
\item Masse kule inspirerende simuleringer p� YouTube. 
\item \verb+http://forum.thegamecreators.com/?m=forum_view&t=167604&b=6+ bruker Verlet og kul grafikk
\item God implementasjon i FORTRAN \verb+http://www.ifa.hawaii.edu/~barnes/ftp/treecode/+
\item sverre.com har mange gode koder (de facto standard vistnok)
\end{itemize}

\end{document}  